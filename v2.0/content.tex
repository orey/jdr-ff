%=======================================
%TODO Frame 1
\begin{frame}[b]

\deuxcolonnesbottom{

\begin{center}
\includegraphics[width=0.4\textwidth]{FF2018.png}
\end{center}

\mysection{Introduction}

\myindent Fighting Fantasy est une série de livres de jeux de rôles pour un seul joueur créée par Steve Jackson (le \href{https://en.wikipedia.org/wiki/Steve_Jackson_(British_game_designer)}{Steve Jackson} anglais et non \href{https://en.wikipedia.org/wiki/Steve_Jackson_(American_game_designer)}{celui de GURPS} qui est américain) et \href{https://en.wikipedia.org/wiki/Ian_Livingstone}{Ian Livingstone}. Le premier volume de la série fut publié en livre de poche par Pufin en 1982.

\myindent Cette série propose un système de jeu de rôles très simple qui fut ensuite complété de diverses façons, notamment dans le livre \textit{Dungeoneer, Advanced Fighting Fantasy}, publié en 1989. Récemment, le système a été repris et customisé dans le jeu de rôles \href{https://melsonian-arts-council.itch.io/troika-numinous-edition}{Troika!}.

\myindent Ce document reprend les règles originales de 1984.

\vspace{0.2cm}

\begin{tabular}{p{4cm}p{4cm}}
Concepteur               & \copyright\ Steve Jackson \\
Version        &  Version originale 1984 \\
Website                  & \href{https://www.fightingfantasy.com/}{www.fightingfantasy.com} \\
Traduction et adaptation & \textcopyleft\ O. Rey 2021-2022 \\
Version                  & \myversion \\
Référence         & \myreference \\ 
\end{tabular}

\mysection{Ce dont vous avez besoin pour jouer}

\myindent Chaque joueur doit avoir accès à :
\begin{myitemize}
\item Un crayon et du papier,
\item Trois dés à 6 faces par joueur,
\item Des choses à picorer et des boissons.
\end{myitemize}

\mysection{Création du personnage}

\mysubsection{Caractéristiques}

\myindent Un personnage possède trois caractéristiques :
\begin{myitemize}
\item Sa \textbf{Compétence} (Skill),
\item Ses \textbf{Points de Vie} (Stamina),
\item Sa \textbf{Chance} (Luck).
\end{myitemize}

\textit{Note} : dans les \textit{Livres Dont Vous Êtes le Héros}, "Skill" est traduit par "Habileté" et "Stamina" par "Constitution". Nous préférons la traduction "Compétence" pour "Skill" car elle plus large que la seule notion d'habileté ; elle est au centre des combats, et couvre les autres compétences. Pour ce qui est de "Stamina", même si la traduction "Constitution" est fidèle, la notion de "Point de Vie" est plus idiomatique du jeu de rôle français.

\myindent Les règles pour déterminer le score dans les différentes caractéristiques sont montrées dans la table suivante :

\begin{center}
\begin{tabular}{lcc}
\textbf{Caractéristique} & \textbf{Création} & \textbf{Acronyme} \\
Compétence        &      1d6+6 & COMP       \\
Points de vie     &     2d6+12 & PV         \\
Chance            &      1d6+6 & CHA        \\
\end{tabular}
\end{center}

\myindent S'il tombe à 0 PV ou moins, le PJ meurt.

\mysubsection{Test de Chance}

\myindent Pour faire un test de Chance, jetez 2d6. Si le résultat est inférieur ou égal à votre CHA, alors c'est réussi. Sinon c'est raté. Dans tous les cas, après le test, enlevez 1 point à votre CHA (-1 CHA).

}{
\mysubsection{Restaurer les caractéristiques}

\vspace{0.1cm}

\begin{tabular}{llp{6cm}}
\textbf{Carac.} & \textbf{Moyen} & \textbf{Effet} \\
COMP       & Potion  & Retour à la valeur initiale \\
PV    & Potion  & Retour à la valeur initiale \\
CHA           & Potion  & Restaure à CHA=CHA+1 (nouvelle valeur initiale) \\
CHA           & MJ      & Bonus de +1 à +3 CHA en fonction des conditions \\
\end{tabular}

\mysubsection{Équipement}

\myindent Tous les personnages démarrent avec les objets suivants :

\vspace{0.2cm}

\begin{tabular}{lcp{5.4cm}}
\textbf{Objet}        & \textbf{Quantité}  & \textbf{Commentaires} \\
Épée       & - & - \\
Sac à dos   & - & Pour mettre les trésors \\
Lanterne   & - & Une lanterne et son combustible suffisent pour une aventure \\
Provisions & 10 repas & Chaque repas (-1 repas) restaure 4 PV \\
               &          & On ne peut pas consommer les provisions pendant un combat \\
Potion     & 2 doses  & Option (choix du MJ) : une potion de Compétence, de Vie ou de Chance (au choix du joueur) \\
               &          & Voir la section \textit{Restaurer les Caractéristiques} \\
\end{tabular}

%=================================================SECTION
\mysection{Monstres}

\mysubsection{Attaques}

\myindent Les montres ont des caractéristiques \textbf{Compétence} et \textbf{Points de Vie}, comme les PJs, mais ils n'ont pas de caractéristique Chance. Ils ont une autre caractéristique : le \textbf{Nombre d'Attaques} (NAT) qui représente le nombre d'adversaires qu'un monstre peut attaquer \textit{en même temps}. Note : tous les PJs ont un NAT de 1.

\mysubsection{Exemple}

\myindent Le loup-garou est un monstre au NAT de 2 : il est donc capable d'attaquer au plus 2 PJs dans un même round de combat.

\begin{center}
\begin{tabular}{lccc}
 & \textbf{COMP} & \textbf{PV} & \textbf{NAT} \\
Loup-garou & 8 & 9 & 2 \\
\end{tabular}
\end{center}

%=================================================SECTION
\mysection{Combat}

\mysubsection{Modificateurs de Compétence}

\myindent Certains objets (comme une épée magique) peuvent apporter des modificateurs au score de COMP.

\myindent Certaines règles s'appliquent :

\begin{myitemize}
\item On ne peut utiliser qu'une seule arme en combat (et donc un seul bonus).
\item La valeur de COMP ne peut jamais excéder sa valeur initiale.
\end{myitemize}

\mysubsection{Combat simple}

\myindent Un round de combat se passe comme suit : les joueurs et les monstres attaquent en même temps en calculant leur score en \textbf{Attaque} (ATT) :
\begin{myitemize}
\item ATT monstre = 2d6 + COMP
\item ATT PJ = 2d6 + COMP
\end{myitemize}

\myindent Les attaques sont ensuite comparées entre elles.

\begin{center}
\begin{tabular}{lll}
\textbf{Comparaison} & \textbf{Effet} & \textbf{Option} \\
ATT PJ $<$ ATT monstre & PJ : -2 PV      & Jet de CHA cas 1 \\
ATT PJ $>$ ATT monstre & Monstre : -2 PV & Jet de CHA cas 2 \\
ATT PJ = ATT monstre & Aucun effet & - \\
\end{tabular}
\end{center}

\myindent Le joueur peut décider d'utiliser sa CHA, soit pour éviter un coup donné par un monstre (cas 1 d'échec du PJ durant le combat), soit pour aggraver la blessure du monstre (cas 2 de succès du PJ durant le combat).

\begin{center}
\begin{tabular}{ll}
\textbf{Jet de CHA} & \textbf{Effet}                                        \\
    Cas 1, réussi          & -1 PV au lieu de -2 PV pour le PJ, -1 CHA      \\
    Cas 1, raté            & -3 PV au lieu de -2 PV pour le PJ, -1 CHA      \\
    Cas 2, réussi          & -4 PV au lieu de -2 PV pour le monstre, -1 CHA \\
    Cas 2, raté            & -1 PV au lieu de -2 PV pour le monstre, -1 CHA \\
\end{tabular}
\end{center}

\myindent Le combat s'arrête quand l'un des deux adversaire est mort (PV=0) ou s'enfuit.

}
\end{frame}

%=======================================
%=======================================
%TODO Frame 2
\begin{frame}[b]

\deuxcolonnesbottom{%col1

\begin{minipage}[c][0.95\textheight][c]{\linewidth}

\myindent Si un PJ s'enfuit, il perd automatiquement 2 PV (dernière blessure infligée par le monstre). La CHA peut être utilisée pour réduire les dommages (jet de CHA, cas 1).

\mysubsection{Combat multiple}

\myindent Le combat multiple est assez amusant et logique. Nous l'exposons au travers d'exemples.

% subsubsection
\textit{\mybullet\ Cas d'un monstre avec un NAT=1 contre 3 PJs (A, B et C)}

\vspace{0.2cm}

\begin{tabular}{cp{7.1cm}}
\textbf{Séquence} & \textbf{Action} \\
         1 & Le MJ tire au sort le PJ qui sera attaqué (ou le choisit), disons C \\
         2 & Combat simple entre le PJ C et le monstre                             \\
           & Le score en ATT du monstre du combat simple, est réutilisé pour les autres combats du même round \\
         3 & Les autres PJ peuvent attaquer le monstre (ici A et B)              \\
           & Si ATT PJ $>$ ATT monstre : -2 PV pour le monstre                     \\
           & Si ATT PJ $<=$ ATT monstre : le monstre n'a rien                      \\
\end{tabular}

\vspace{0.2cm}

\myindent A et B ne peuvent pas prendre de dommages (perte de PV). On appelle leurs attaques des \textbf{attaques protégées}.

\myindent Au round suivant, le processus recommence.

\vspace{0.2cm}

% subsubsection
\textit{\mybullet\ Cas d'un monstre avec un NAT=8 contre 4 PJs (A, B, C, D)}

\textit{Note} : même si le NAT du monstre est supérieur au nombre de PJs, ce dernier n'a pas d'attaques supplémentaires. Le NAT correspond au nombre maximum de PJs que le monstre peut attaquer. Dans le cas présent, le monstre ne pourra attaquer que les 4 PJs, 1 fois par round.

\vspace{0.2cm}

\begin{tabular}{cp{7.1cm}}
\textbf{Séquence} & \textbf{Action} \\
         1 & Le MJ calcule le score d'ATT du monstre                         \\
           & Ce nombre est valable pour le round \textit{pour tous les combats avec tous les PJs} \\
         2 & Chaque combat est un combat simple résolu normalement              \\
\end{tabular}

\vspace{0.2cm}

% subsubsection
\textit{\mybullet\ Cas de deux PJs (A et B) contre deux monstres (X NAT=2, Y NAT=1)}

\vspace{0.2cm}

\begin{tabular}{cp{7.1cm}}
\textbf{Séquence} & \textbf{Action} \\
         1 & Le MJ demande aux joueurs quels monstres ils veulent attaquer (ex: A-X et B-Y) \\
         2 & Combats simples entre A et X, et B et Y                   \\
         3 & X a une seconde attaque, il peut donc attaquer B en mode \textit{attaque protégée} \\
\end{tabular}

\vspace{0.2cm}

\myindent Tout monstre supplémentaire attaquera de manière aléatoire l'un des deux PJs.

%======================= SECTION
\mysection{Situations communes}

\mysubsection{Équipement des PJs}

\myindent Les PJs ne peuvent pas transporter un nombre illimité de choses. Un PJ ne devrait pas transporter plus de 10 articles d'équipement (hors or et provisions). Les gros objets comptent pour plus d'un point. Le MJ doit être vigilant sur ce point.

\mysubsection{Portes}

\vspace{0.2cm}

\begin{tabular}{>{\raggedright}p{1.8cm}p{6.2cm}}
\textbf{Type} & \textbf{Commentaire} \\
Porte magique              & Ont besoin d'un sort pour être ouvertes (ou sous contrôle d'un sorcier) \\
Porte ordinaire            & Jeter 1d6 : 1-2 la porte est fermée ; 3-6 la porte est ouverte          \\
Casser une porte ordinaire & Jet réussi de 2d6 strictement sous COMP ; -1 PV                         \\
                           & Si le jet est supérieur ou égal à COMP, la porte résiste ; -1 PV        \\
                           & Deuxième tentative : 2d6 + 1 strictement sous COMP pour réussir ; -1 PV \\
                           & Troisième tentative : 2d6 + 2 strictement... (etc.)                     \\
Portes secrètes            & Le PJ doit chercher ; le MJ jette 2d6 sous la COMP du PJ                \\
                           & Si le jet est réussi, la porte est trouvée (mais pas ouverte)           \\
                           & Jet de CHA pour trouver comment l'ouvrir                                \\
\end{tabular}

\mysubsection{Perte d'un arme}

\myindent Si un PJ perd son arme, sa COMP est diminuée de 4 jusqu'à ce qu'il trouve une autre arme.

\end{minipage}
}
{
\begin{minipage}[c][0.95\textheight][c]{\linewidth}

\mysubsection{Fuite}

\myindent Le MJ doit décider si la fuite est possible (par exemple PJ acculé). Si la fuite est possible, la règle ci-dessus (combat simple) s'applique. Idem pour les monstres (intelligents) qui fuient.

\mysubsection{Soudoyer/corrompre}

\myindent Les monstres un peu intelligents aiment l'or. Le MJ peut accepter que les PJs tentent de les corrompre. Le MJ décide d'une probabilité de réussite et lance 1d6 (1 sur 6, ou 3 sur 6, etc.). Les monstres peuvent donner quelques informations s'ils se font corrompre.

\mysubsection{Chute}

\vspace{0.2cm}

\begin{tabular}{lp{5.8cm}}
\textbf{Hauteur} & \textbf{Commentaire} \\
Inférieur à 2m     & Pas de dommages                                                   \\
Par tranche de 10m & Faire un jet 2d6 + 1 sous CHA                                     \\
                   & Ex : 10m, 2d6 + 1 sous CHA ; 30m, 2d6+3 sous CHA                  \\
                   & Si jet de CHA raté, le PJ est blessé. Perte de PV : 1 + 1 par 10m \\
\end{tabular}

\vspace{0.2cm}

\mysubsection{Mouvement}

\myindent Laissé à l'arbitrage du MJ et suivant les situations (longs couloirs avec pièges).

\mysubsection{Ouvrir un coffre}

\vspace{0.2cm}

\begin{tabular}{>{\raggedright}p{2cm}p{6cm}}
État du coffre         & Jeter 1d6 : 1-5, le coffre est fermé ; 6, il est ouvert \\
Ouvrir un coffre fermé & 2d6 strictement sous COMP \\
                       & En cas d'échec, si le PJ retente, à chaque essai, son arme s'abîme et le PJ perd un point de COMP par tentative jusqu'à ce qu'il trouve une autre arme \\
Compartiments secrets & Le PJ doit chercher le compartiment. La règle des portes secrètes s'applique \\
\end{tabular}

\vspace{0.2cm}

\mysubsection{Provisions}

\myindent Les PJ peuvent consommer leurs provisions à tout moment sauf dans un combat.

\myindent Le nombre de provisions dont bénéficient les PJs au départ de l'aventure dépend de différents facteurs : longueur de l'histoire, provisions disponibles dans le scénario, etc.

\begin{wraptable}{r}{3.2cm}
\begin{tabular}{lc}
\textbf{Aventure} & \textbf{Provisions} \\
Courte & 2 \\
Moyenne & 4 \\
Longue & 6+ \\
\end{tabular}
\end{wraptable}

\mysubsection{Chercher}

\myindent Le PJ doit dire ce qu'il cherche. Le MJ fait les jets de dés : 2d6 strictement sous COMP pour trouver.

\mysubsection{Se déplacer en silence}

\myindent 2d6 strictement sous COMP. Le MJ peut ajouter des malus.

\mysubsection{Pickpocket}

\myindent Un jet de COMP strictement réussi est un succès. Le MJ peut donner un malus (6-8 est un malus acceptable si la situation ne se prête pas à jouer au pickpocket).

\mysubsection{Monstres errants}

\myindent Si les PJs s'attardent trop dans un lieu, il est possible de générer une rencontre avec un monstre errant. le MJ lance 1d6 régulièrement. Si c'est un 1, un monstre a repéré les PJs.

\myindent En souterrain :

\begin{center}
\begin{tabular}{clccc}
\textbf{1d6} &  \textbf{Créature} &  \textbf{COMP} &  \textbf{PV} &  \textbf{NAT} \\
    1 & Goblin     &      5 &    3 &     1 \\
    2 & Orc        &      6 &    3 &     1 \\
    3 & Gremlin    &      6 &    3 &     1 \\
    4 & Rat géant  &      5 &    4 &     1 \\
    5 & Squelette  &      6 &    5 &     1 \\
    6 & Troll      &      8 &    7 &     3 \\
\end{tabular}
\end{center}

\myindent En extérieur :

\begin{center}
\begin{tabular}{clccc}
\textbf{1d6} &  \textbf{Créature} &  \textbf{COMP} &  \textbf{PV} &  \textbf{NAT} \\
    1 &  Goblin               &       5 &     3 &      1 \\
    2 &  Chauve-souris géante &       5 &     4 &      1 \\
    3 &  Rat Géant            &       5 &     4 &      1 \\
    4 &  Chien de guerre      &       7 &     6 &      1 \\
    5 &  Loup-garou           &       8 &     9 &      2 \\
    6 &  Ogre                 &       8 &    10 &      2 \\
\end{tabular}
\end{center}

\end{minipage}
}
\end{frame}

\begin{comment}
%=======================================
\begin{frame}[t]

\deuxcolonnes{%col1

}{

}
\end{frame}


\end{comment}

